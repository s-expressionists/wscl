% -*- Mode: TeX -*-

% Strings

%-------------------- Strings --------------------

%% 2.5.2 1
%% 18.0.0 3
\begincom{string}\ftype{System Class}

\label Class Precedence List::
\typeref{string},
%% 2.15.0 16
\typeref{vector},
\typeref{array},
\typeref{sequence},
\typeref{t}

\label Description::

\issue{CHARACTER-PROPOSAL:2-3-2}
A \term{string} is a \term{specialized} \term{vector} 
whose \term{elements} are \oftypes{character}.  
When used as a \term{type specifier} for object creation,
\typeref{string} means \f{(vector character)}.
\endissue{CHARACTER-PROPOSAL:2-3-2}

\issue{CHARACTER-PROPOSAL:2-3-3}

\label Compound Type Specifier Kind::

Abbreviating.

\label Compound Type Specifier Syntax::

%% 4.6.0 10
\Deftype{string}{\ttbrac{size}}

\label Compound Type Specifier Arguments::

\issue{ARRAY-DIMENSION-IMPLICATIONS:ALL-FIXNUM}
\param{size}---a non-negative \term{fixnum},
	    or the \term{symbol} \misc{*}.
\endissue{ARRAY-DIMENSION-IMPLICATIONS:ALL-FIXNUM}

\label Compound Type Specifier Description::
         
This denotes the union of all \term{types}
\f{(array \i{c} (\param{size}))}
for all \term{subtypes} \i{c} of \typeref{character};
that is, the set of \term{strings} of size \param{size}.

\label See Also::

{\secref\StringConcepts},
{\secref\Doublequote},
{\secref\PrintingStrings}

\endcom%{string}\ftype{System Class}
\begincom{base-string}\ftype{Type}

\issue{CHARACTER-PROPOSAL:2-3-3}

\label Supertypes::

\typeref{base-string},
\typeref{string},
\typeref{vector},
\typeref{array},
\typeref{sequence},
\typeref{t}

\label Description::

\Thetype{base-string} is equivalent to 
\issue{CHARACTER-VS-CHAR:LESS-INCONSISTENT-SHORT}
\f{(vector base-char)}.
\endissue{CHARACTER-VS-CHAR:LESS-INCONSISTENT-SHORT}
% A \term{base string} is the most efficient \term{string} that can hold
% %%this next phrase added in answer to Symbolics' concerns that
% %%otherwise a special representation for certain subsets of standard-char
% %%might be unintentionally prohibited by the spec.
% an arbitrary sequence of \term{standard characters}.
%% Rewritten for Sandra to use "representation" -kmp 13-Jan-92
%!!! KAB (re: "efficient") I think we never decided what this meant. True?
%KMP: True.
The \term{base string} representation is the most efficient \term{string} representation
that can hold an arbitrary sequence of \term{standard characters}.

\endissue{CHARACTER-PROPOSAL:2-3-3}
%!!! Should this be said somewhere??
%
%\issue{CHARACTER-PROPOSAL:2-3}
%An implementation can support \term{string} subtypes in addition
%to the ones defined in this standard.
%For example, an
%implementation supporting Arabic and Cyrillic characters
%might provide as extended characters:
%\beginlist
%\itemitem{\typeref{general-string}}
%
%Can contain Arabic, Cyrillic or \term{base characters}
%in any mixture.
%
%\itemitem{\typeref{region-specialized-string}}
%
%Can contain installation-selected repertoire 
%(Arabic/Cyrillic) or \term{base characters} in any mixture.
%
%\itemitem{\typeref{base-string}}
%
%Can contain \term{base characters}.
%\endlist
%Though portability of applications using
%\typeref{region-specialized-string} is limited, a performance
%advantage might argue for its use.
% 
%Alternatively,
%an implementation
%supporting a large base character repertoire
%including, say, Japanese Kanji might define
%\issue{CHARACTER-VS-CHAR:LESS-INCONSISTENT-SHORT}
%\typeref{base-char}
%\endissue{CHARACTER-VS-CHAR:LESS-INCONSISTENT-SHORT}
%as equivalent to \term{character}.
% %We expect that applications sensitive to the performance
%%of character handling in some host environments will
%%utilize the string subtypes to provide performance
%%improvement.  Applications with emphasis on international
%%portability will likely utilize only {\clkwd general-strings}.
%\term{Base strings} allow for more compact representation of 
%\term{strings}
%of \term{base characters}, which are likely to predominate in any system.
%Note that in any particular implementation the \term{base characters}
%need not be the
%most compactly representable, since others might have
%a smaller repertoire.
%However, in most implementations \term{base strings} are
%likely to be more space efficient than other types of \term{strings}.
%\endissue{CHARACTER-PROPOSAL:2-3}

\label Compound Type Specifier Kind::

Abbreviating.

\label Compound Type Specifier Syntax::

\issue{JUN90-TRIVIAL-ISSUES:24}
\Deftype{base-string}{\ttbrac{size}}
\endissue{JUN90-TRIVIAL-ISSUES:24}

\label Compound Type Specifier Arguments::

\issue{ARRAY-DIMENSION-IMPLICATIONS:ALL-FIXNUM}
\param{size}---a non-negative \term{fixnum},
	    or the \term{symbol} \misc{*}.
\endissue{ARRAY-DIMENSION-IMPLICATIONS:ALL-FIXNUM}

\label Compound Type Specifier Description::

This is equivalent to the type \f{(vector base-char \param{size})};
that is, the set of \term{base strings} of size \param{size}.

\endissue{CHARACTER-PROPOSAL:2-3-3}
\endcom%{base-string}\ftype{Type}
\begincom{simple-string}\ftype{Type}

\label Supertypes:: 

\typeref{simple-string},
%% 2.15.0 23
\typeref{string},
\typeref{vector},
\typeref{simple-array},
\typeref{array},
\typeref{sequence},
\typeref{t}

\label Description::

\issue{CHARACTER-PROPOSAL:2-3-4}
%A \term{string} that is not displaced to another \term{array},
%has no \term{fill pointer}, and is not to have its size adjusted 
%dynamically after creation is a \typeref{simple-string}.
A \term{simple string} is a specialized one-dimensional
\term{simple array} whose \term{elements} are \oftypes{character}.
When used as a \term{type specifier} for object creation,
\typeref{simple-string} means \f{(simple-array character (\i{size}))}.
\endissue{CHARACTER-PROPOSAL:2-3-4}

\label Compound Type Specifier Kind::

Abbreviating.

\label Compound Type Specifier Syntax::

\Deftype{simple-string}{\ttbrac{size}}

\label Compound Type Specifier Arguments::

\issue{ARRAY-DIMENSION-IMPLICATIONS:ALL-FIXNUM}
\param{size}---a non-negative \term{fixnum},
	    or the \term{symbol} \misc{*}.
\endissue{ARRAY-DIMENSION-IMPLICATIONS:ALL-FIXNUM}

\label Compound Type Specifier Description::

%% 4.6.0 11                                  
%\itemitem{\tt (simple-string \param{size})}

This denotes the union of all \term{types}
{\tt (simple-array \i{c} (\param{size}))} for all \term{subtypes} \i{c} of
\typeref{character}; that is, the set of \term{simple strings} of size \param{size}.

\endcom%{simple-string}\ftype{Type}
\issue{CHARACTER-PROPOSAL:2-3-5}
\begincom{simple-base-string}\ftype{Type}

\label Supertypes::

\typeref{simple-base-string},
\typeref{base-string},
\typeref{simple-string},
\typeref{string},
\typeref{vector},
\typeref{simple-array},
\typeref{array},
\typeref{sequence},
\typeref{t}

\label Description::

\Thetype{simple-base-string} is equivalent to 
\issue{CHARACTER-VS-CHAR:LESS-INCONSISTENT-SHORT}
\f{(simple-array base-char (*))}.
\endissue{CHARACTER-VS-CHAR:LESS-INCONSISTENT-SHORT}
\endissue{CHARACTER-PROPOSAL:2-3-5}

\label Compound Type Specifier Kind::

Abbreviating.

\label Compound Type Specifier Syntax::

\issue{JUN90-TRIVIAL-ISSUES:24}
\Deftype{simple-base-string}{\ttbrac{size}}
\endissue{JUN90-TRIVIAL-ISSUES:24}

\label Compound Type Specifier Arguments::

\issue{ARRAY-DIMENSION-IMPLICATIONS:ALL-FIXNUM}
\param{size}---a non-negative \term{fixnum},
	    or the \term{symbol} \misc{*}.
\endissue{ARRAY-DIMENSION-IMPLICATIONS:ALL-FIXNUM}

\label Compound Type Specifier Description::

This is equivalent to the type \f{(simple-array base-char (\param{size}))};
that is, the set of \term{simple base strings} of size \param{size}.

\endcom%{simple-base-string}\ftype{Type}


%%% ========== SIMPLE-STRING-P
\begincom{simple-string-p}\ftype{Function}

\label Syntax::

\DefunWithValues simple-string-p {object} {generalized-boolean}

\label Arguments and Values::

\param{object}---an \term{object}.

\param{generalized-boolean}---a \term{generalized boolean}.

\label Description::

%% 6.2.2 22
\TypePredicate{object}{simple-string}

\label Examples::
\code
 (simple-string-p "aaaaaa") \EV \term{true}
 (simple-string-p (make-array 6 
                              :element-type 'character 
                              :fill-pointer t)) \EV \term{false}
\endcode

\label Side Effects:\None.

\label Affected By:\None.

\label Exceptional Situations:\None!

\label See Also:\None.

\label Notes::
\code
 (simple-string-p \param{object}) \EQ (typep \param{object} 'simple-string)
\endcode

\endcom

%%% ========== CHAR
%%% ========== SCHAR
\begincom{char, schar}\ftype{Accessor}

\label Syntax::

\DefunMultiWithValues {string index} {character}
 {\entry{char}
  \entry{schar}}

\DefsetfMulti {string index} {new-character}
 {\entry{char}
  \entry{schar}}

\label Arguments and Values:: 

%% 18.1.0 4
\param{string}---for \funref{char},  a \term{string};
		 for \funref{schar}, a \term{simple string}.

\param{index}---a \term{valid array index} for the \param{string}.

\param{character}, \param{new-character}---a \term{character}.
       
\label Description::

%% 18.1.0 3
\funref{char} and \funref{schar} \term{access} the \term{element} of \param{string}
specified by \param{index}.
%% This is now said in the concept stuff.
% \param{Index} is an offset from the beginning
% of \param{string}; indexing is zero-origin.

\funref{char} ignores \term{fill pointers} when \term{accessing} \term{elements}.

%% KAB: Implied by "access".
% \macref{setf} can be used with \funref{char} or \funref{schar}
% to destructively replace an \term{element} of the \param{string}.

\label Examples::

\code
 (setq my-simple-string (make-string 6 :initial-element #\\A)) \EV "AAAAAA"
 (schar my-simple-string 4) \EV #\\A
 (setf (schar my-simple-string 4) #\\B) \EV #\\B
 my-simple-string \EV "AAAABA"
 (setq my-filled-string
       (make-array 6 :element-type 'character
                     :fill-pointer 5
                     :initial-contents my-simple-string))
\EV "AAAAB"
 (char my-filled-string 4) \EV #\\B
 (char my-filled-string 5) \EV #\\A
 (setf (char my-filled-string 3) #\\C) \EV #\\C
 (setf (char my-filled-string 5) #\\D) \EV #\\D
 (setf (fill-pointer my-filled-string) 6) \EV 6
 my-filled-string \EV "AAACBD"
\endcode

\label Affected By:\None.

\label Exceptional Situations:

%!!! Barrett thinks stuff belongs here.
\issue{CHAR-SCHAR-TYPE-ERROR:SIGNAL-ERROR-IN-SAFE-CODE}
\None.
\endissue{CHAR-SCHAR-TYPE-ERROR:SIGNAL-ERROR-IN-SAFE-CODE}

\label See Also::

\funref{aref},
\funref{elt},
\issue{CONSTANT-MODIFICATION:DISALLOW}
{\secref\ConstantModification}
\endissue{CONSTANT-MODIFICATION:DISALLOW}

\label Notes::

\code
 (char s j) \EQ (aref (the string s) j)
\endcode

\endcom

%%% ========== STRING
\begincom{string}\ftype{Function}

\label Syntax::

\DefunWithValues string {x} {string}

\label Arguments and Values::

\issue{CHARACTER-PROPOSAL:2-1-1}
\param{x}---a \term{string}, a \term{symbol}, or a \term{character}.
\endissue{CHARACTER-PROPOSAL:2-1-1}

\param{string}---a \term{string}.

\label Description::

Returns a \term{string} described by \param{x}; specifically:

\beginlist
\item{\bull} If \param{x} is a \term{string}, it is returned.
\item{\bull} If \param{x} is a \term{symbol}, its \term{name} is returned.
\item{\bull} 
\issue{CHARACTER-PROPOSAL:2-1-1}%
If \param{x} is a \term{character},
\endissue{CHARACTER-PROPOSAL:2-1-1}
then a \term{string} containing that one \term{character} is returned.
\item{\bull}
\issue{STRING-COERCION:MAKE-CONSISTENT}%
\funref{string} might perform additional, \term{implementation-defined} conversions.
\endissue{STRING-COERCION:MAKE-CONSISTENT}
\endlist

\label Examples::

\code
 (string "already a string") \EV "already a string"
 (string 'elm) \EV "ELM"
 (string #\\c) \EV "c"
\endcode

\label Affected By:\None.

\label Exceptional Situations::

\issue{STRING-COERCION:MAKE-CONSISTENT}
In the case where a conversion is defined neither by this specification nor
by the \term{implementation}, an error \oftype{type-error} is signaled.
\endissue{STRING-COERCION:MAKE-CONSISTENT}

\label See Also::

\funref{coerce},
\typeref{string} (\term{type}).

\label Notes::

%% 18.3.0 14
%% KMP (13-Feb-91): Need to make it so that this is apparent and needn't be inferred
%%    from weird rules of thumb.
%% Barmar: Do they allow a character argument?
%% KMP: No! Hmm. Removed for now as misleading.
% Most of the string functions effectively apply \funref{string}
% to their arguments that are supposed to be \term{strings}.

%% 18.3.0 15
\funref{coerce} can be used to convert a \term{sequence} of \term{characters}
to a \term{string}.

%% 18.3.0 15
\funref{prin1-to-string}, \funref{princ-to-string}, \funref{write-to-string},
or \funref{format} (with a first argument of \nil) can be used to get a
\term{string} representation of a \term{number} or any other \term{object}.

\endcom

%%% ========== NSTRING-CAPITALIZE
%%% ========== NSTRING-DOWNCASE
%%% ========== NSTRING-UPCASE
%%% ========== STRING-CAPITALIZE
%%% ========== STRING-DOWNCASE
%%% ========== STRING-UPCASE
\begincom{string-upcase, string-downcase, string-capitalize, 
nstring-upcase, nstring-downcase, nstring-capitalize}\ftype{Function}


\label Syntax::

\DefunMultiWithValues {string {\key} start end} {cased-string}
 {\entry{string-upcase}
  \entry{string-downcase}
  \entry{string-capitalize}}

\DefunMultiWithValues {string {\key} start end} {string}
 {\entry{nstring-upcase}
  \entry{nstring-downcase}
  \entry{nstring-capitalize}}

\label Arguments and Values:: 

\issue{STRING-COERCION:MAKE-CONSISTENT}
\param{string}---a \term{string designator}.
  For \funref{nstring-upcase},
      \funref{nstring-downcase},
  and \funref{nstring-capitalize},
  the \param{string} \term{designator} must be a \term{string}.
\endissue{STRING-COERCION:MAKE-CONSISTENT}

\issue{SUBSEQ-OUT-OF-BOUNDS}                       
\issue{RANGE-OF-START-AND-END-PARAMETERS:INTEGER-AND-INTEGER-NIL}
\param{start}, \param{end}---\term{bounding index designators} of \param{string}.
 \Defaults{\param{start} and \param{end}}{\f{0} and \nil}
\endissue{RANGE-OF-START-AND-END-PARAMETERS:INTEGER-AND-INTEGER-NIL}
\endissue{SUBSEQ-OUT-OF-BOUNDS}

%% 18.3.0 13
\param{cased-string}---a \term{string}.

\label Description::

\funref{string-upcase},  \funref{string-downcase},  \funref{string-capitalize},
\funref{nstring-upcase}, \funref{nstring-downcase}, \funref{nstring-capitalize}
change the case of the subsequence of \param{string} 
\term{bounded} by \param{start} and \param{end} 
as follows:

\beginlist
\itemitem{\bf string-upcase} 

%% 18.3.0 6
\funref{string-upcase} returns a \term{string} just like \param{string} 
with all lowercase characters replaced by the corresponding uppercase 
characters.  More precisely, each character of the result \term{string} 
is produced by applying \thefunction{char-upcase} to the corresponding
character of \param{string}.

\itemitem{\bf string-downcase} 

%% 18.3.0 7
\funref{string-downcase} is like \funref{string-upcase} 
except that all uppercase characters are replaced by the corresponding 
lowercase characters (using \funref{char-downcase}).


\itemitem{\bf string-capitalize}

%% 18.3.0 10
\funref{string-capitalize} produces a copy of \param{string} such that,
for every word in the copy, the first \term{character} of the ``word,''
if it has \term{case}, is \term{uppercase} and
any other \term{characters} with \term{case} in the word are \term{lowercase}.
For the purposes of \funref{string-capitalize},
a ``word'' is defined to be a
%% Rewritten on advice of Barmar, with confirmation from Quinquevirate.
%consecutive subsequence consisting of case-modifiable characters or digits,
consecutive subsequence consisting of \term{alphanumeric} \term{characters},
delimited at each end either by a non-\term{alphanumeric} \term{character}
or by an end of the \term{string}.

%% 18.3.0 12
\itemitem{\bf nstring-upcase, nstring-downcase, nstring-capitalize }  

%Barrett wondered if these really had to modify the string. CLtL p304 says yes. -kmp 13-Feb-92
\funref{nstring-upcase}, \funref{nstring-downcase},
and \funref{nstring-capitalize} are identical to \funref{string-upcase}, 
\funref{string-downcase}, and \funref{string-capitalize}
respectively except that they  modify \param{string}.
\endlist

%% Barrett: Duplication
% \param{start} marks the beginning position of the substring.
% \param{end}   marks the position following the last element of the substring.

%% 18.3.0 9
For \funref{string-upcase}, \funref{string-downcase}, and \funref{string-capitalize},
\param{string} is not modified.  However, if no characters in \param{string} 
require conversion, the result may be either \param{string} or a copy of it,
at the implementation's discretion.

%% Implied by use of "string designator".
% \issue{STRING-COERCION:MAKE-CONSISTENT}
% \funref{string-upcase}, \funref{string-downcase}, and \funref{string-capitalize}
% perform coercion identical to the action of \funref{string}. 
% \endissue{STRING-COERCION:MAKE-CONSISTENT}

\label Examples::
\code
 (string-upcase "abcde") \EV "ABCDE"
 (string-upcase "Dr. Livingston, I presume?")
\EV "DR. LIVINGSTON, I PRESUME?"
 (string-upcase "Dr. Livingston, I presume?" :start 6 :end 10)
\EV "Dr. LiVINGston, I presume?"
 (string-downcase "Dr. Livingston, I presume?")
\EV "dr. livingston, i presume?"

 (string-capitalize "elm 13c arthur;fig don't") \EV "Elm 13c Arthur;Fig Don'T"
 (string-capitalize " hello ") \EV " Hello "
 (string-capitalize "occlUDeD cASEmenTs FOreSTAll iNADVertent DEFenestraTION")
\EV  "Occluded Casements Forestall Inadvertent Defenestration"
 (string-capitalize 'kludgy-hash-search) \EV "Kludgy-Hash-Search"
 (string-capitalize "DON'T!") \EV "Don'T!"    ;not "Don't!"
 (string-capitalize "pipe 13a, foo16c") \EV "Pipe 13a, Foo16c"

 (setq str (copy-seq "0123ABCD890a")) \EV "0123ABCD890a"
 (nstring-downcase str :start 5 :end 7) \EV "0123AbcD890a"
 str \EV "0123AbcD890a"
\endcode


\label Side Effects::

    \funref{nstring-upcase},
    \funref{nstring-downcase}, 
and \funref{nstring-capitalize} modify \param{string} as appropriate
rather than constructing a new \term{string}.

\label Affected By:\None.

%% Sandra thinks this is too vague.
%The \term{implementation}.

\label Exceptional Situations:\None.

\label See Also::

\funref{char-upcase}, \funref{char-downcase}

\label Notes::
%% 18.3.0 8
The result is always of the same length
as \param{string}.

\endcom

%%% ========== STRING-TRIM
%%% ========== STRING-LEFT-TRIM
%%% ========== STRING-RIGHT-TRIM
\begincom{string-trim, string-left-trim, string-right-trim}\ftype{Function}

\label Syntax::
\DefunMultiWithValues {character-bag string} {trimmed-string}
  {\entry{string-trim}
   \entry{string-left-trim}
   \entry{string-right-trim}}

\label Arguments and Values::

\param{character-bag}---a \term{sequence} containing \term{characters}.

\issue{STRING-COERCION:MAKE-CONSISTENT}
\param{string}---a \term{string designator}.
\endissue{STRING-COERCION:MAKE-CONSISTENT}

\param{trimmed-string}---a \term{string}.

\label Description::

%% 18.3.0 5
\funref{string-trim} returns a substring of \param{string}, 
with all characters in \param{character-bag} stripped off the beginning and end.
\funref{string-left-trim} is similar but strips characters off only the beginning; 
\funref{string-right-trim} strips off only the end.

If no \term{characters} need to be trimmed from the \param{string},
then either \param{string} itself or a copy of it may be returned,
at the discretion of the implementation.

%Added for Barrett.
All of these \term{functions} observe the \term{fill pointer}.

%% Implied by "string designator".
% \issue{STRING-COERCION:MAKE-CONSISTENT}
% The functions herein described perform coercion identical to the action of \funref{string}. 
% \endissue{STRING-COERCION:MAKE-CONSISTENT}

\label Examples::
\code
 (string-trim "abc" "abcaakaaakabcaaa") \EV "kaaak"
 (string-trim '(#\\Space #\\Tab #\\Newline) " garbanzo beans
        ") \EV "garbanzo beans"
 (string-trim " (*)" " ( *three (silly) words* ) ")
\EV "three (silly) words"

 (string-left-trim "abc" "labcabcabc") \EV "labcabcabc"
 (string-left-trim " (*)" " ( *three (silly) words* ) ")
\EV "three (silly) words* ) "

 (string-right-trim " (*)" " ( *three (silly) words* ) ") 
\EV " ( *three (silly) words"
\endcode
\label Side Effects:\None.

\label Affected By::

The \term{implementation}.

\label Exceptional Situations:\None.

\label See Also:\None.

\label Notes:\None.

\endcom

%%% ========== STRING=
%%% ========== STRING/=
%%% ========== STRING<
%%% ========== STRING>
%%% ========== STRING>=
%%% ========== STRING<=
%%% ========== STRING-EQUAL
%%% ========== STRING-NOT-EQUAL
%%% ========== STRING-GREATERP
%%% ========== STRING-LESSP
%%% ========== STRING-NOT-LESSP
%%% ========== STRING-NOT-GREATERP
\begincom{string=, string/=, string<, string>, string<=, string>=, 
	  string-equal, string-not-equal, string-lessp, 
	  string-greaterp, string-not-greaterp, string-not-lessp}\ftype{Function}

\label Syntax::

%!!! Need to cross-check the return values here against CLtL for accuracy.
%    My guess is that something's been corrupted along the way. -kmp 27-May-91

\DefunWithValues {string$=$}
		 {string1 string2 {\key} start1 end1 start2 end2}
		 {generalized-boolean}

\DefunMultiWithValues {string1 string2 {\key} start1 end1 start2 end2}
		      {mismatch-index}
  {\entry{string/$=$}
   \entry{string$<$}
   \entry{string$>$}
   \entry{string$<=$}
   \entry{string$>=$}}

\DefunWithValues {string-equal}
		 {string1 string2 {\key} start1 end1 start2 end2}
		 {generalized-boolean}

\DefunMultiWithValues {string1 string2 {\key} start1 end1 start2 end2}
		      {mismatch-index}
  {\entry{string-not-equal}
   \entry{string-lessp}
   \entry{string-greaterp}
   \entry{string-not-greaterp}
   \entry{string-not-lessp}}

\label Arguments and Values::

%% 18.2.0 4
%% 18.2.0 11
\issue{STRING-COERCION:MAKE-CONSISTENT}
\param{string1}---a \term{string designator}.
                
\param{string2}---a \term{string designator}.
\endissue{STRING-COERCION:MAKE-CONSISTENT}

\issue{SUBSEQ-OUT-OF-BOUNDS}
\issue{RANGE-OF-START-AND-END-PARAMETERS:INTEGER-AND-INTEGER-NIL}
\param{start1}, \param{end1}---\term{bounding index designators} of \param{string1}.
 \Defaults{\param{start} and \param{end}}{\f{0} and \nil}

\param{start2}, \param{end2}---\term{bounding index designators} of \param{string2}.
 \Defaults{\param{start} and \param{end}}{\f{0} and \nil}
\endissue{RANGE-OF-START-AND-END-PARAMETERS:INTEGER-AND-INTEGER-NIL}
\endissue{SUBSEQ-OUT-OF-BOUNDS}

\param{generalized-boolean}---a \term{generalized boolean}.

\param{mismatch-index}---a \term{bounding index} of \param{string1}, or \nil.

\label Description::

%% 18.2.0 9
These functions perform lexicographic comparisons on \param{string1} and \param{string2}.
\funref{string=} and \funref{string-equal} are called equality functions;
the others are called inequality functions.
The comparison operations these \term{functions} perform are restricted
     to the subsequence of \param{string1} \term{bounded} by \term{start1} and \param{end1}
 and to the subsequence of \param{string2} \term{bounded} by \term{start2} and \param{end2}.
               
A string \i{a} is equal to a string \i{b} if it contains the same number
of characters, and the corresponding characters are the \term{same}
% "char<", "char-lessp" => "char=", "char-equal" in next line thanks to Barmar! -kmp 22-Jan-92
under \funref{char=} or \funref{char-equal}, as appropriate.

%% 18.2.0 10
A string \i{a} is less than a string \i{b} if in the first position in
which they differ the character of \i{a} is less than the corresponding
character of \i{b} according to \funref{char<} or \funref{char-lessp} 
as appropriate, or if string \i{a} is a proper prefix of string \i{b}
(of shorter length and matching in all the characters of \i{a}).

%Sandra is a little uncomfortable with the use of "equal" in the next two paragraphs.
% -kmp 13-Jan-92
The equality functions return a \param{generalized boolean} 
that is \term{true} if the strings are equal, 
or \term{false} otherwise.

The inequality functions return a \param{mismatch-index}
that is \term{true} if the strings are not equal,
or \term{false} otherwise.
When the \param{mismatch-index} is \term{true},
it is an \term{integer} representing the first character position at which the 
two substrings differ, as an offset from the beginning of \param{string1}.

%% This referred only to string= and string-equal, but was confusing and redundant.
%%   -kmp 25-Jul-91
% %% 18.2.0 5
% The equality functions return \term{false} if the substrings being compared are
% of unequal length; that is, if
% 
% \code
%  (not (= (- end1 start1) (- end2 start2)))
% \endcode
% is \term{true}, then the functions return \term{false}.

The comparison has one of the following results:

\beginlist
     
\itemitem{\funref{string=}}

\funref{string=} is \term{true} if the supplied substrings are of
the same length and contain the \term{same} characters in corresponding
positions; otherwise it is \term{false}.

\itemitem{\funref{string/=}}

\funref{string/=} is \term{true} if the supplied substrings are 
different; otherwise it is \term{false}.

\itemitem{\funref{string-equal}}

%% 18.2.0 8
\funref{string-equal} is just like \funref{string=} 
except that differences in case are ignored; 
two characters are considered to be the same if \funref{char-equal} is \term{true} of them.
     
\itemitem{\funref{string<}}

\funref{string<} is \term{true} if substring1 is less than substring2;
otherwise it is \term{false}.

\itemitem{\funref{string>}}

\funref{string>} is \term{true} if substring1 is greater than substring2;
otherwise it is \term{false}.

\itemitem{\funref{string-lessp}, \funref{string-greaterp}}

%% 18.2.0 12
\funref{string-lessp} and \funref{string-greaterp} 
are exactly like \funref{string<} and \funref{string>}, respectively,
except that distinctions between uppercase and lowercase letters are ignored.
It is as if \funref{char-lessp} were used instead of \funref{char<}
for comparing characters.

\itemitem{\funref{string<=}}

\funref{string<=} is \term{true} if substring1 is less than or equal to substring2;
otherwise it is \term{false}.

\itemitem{\funref{string>=}}

\funref{string>=} is \term{true} if substring1 is greater than or equal to substring2;
otherwise it is \term{false}.

\itemitem{\funref{string-not-greaterp}, \funref{string-not-lessp}}

\funref{string-not-greaterp} and \funref{string-not-lessp}
are exactly like \funref{string<=} and \funref{string>=}, respectively, 
except that distinctions between uppercase and lowercase letters are ignored.
It is as if \funref{char-lessp} were used instead of \funref{char<}
for comparing characters.

\endlist

%% Barrett: Implied by description of bounding.
%All of these \term{functions} observe the \term{fill pointer}.
 
\label Examples::

%% 18.2.0 6

\code
 (string= "foo" "foo") \EV \term{true}
 (string= "foo" "Foo") \EV \term{false}
 (string= "foo" "bar") \EV \term{false}
 (string= "together" "frog" :start1 1 :end1 3 :start2 2) \EV \term{true}
 (string-equal "foo" "Foo") \EV \term{true}
 (string= "abcd" "01234abcd9012" :start2 5 :end2 9) \EV \term{true}
 (string< "aaaa" "aaab") \EV 3
 (string>= "aaaaa" "aaaa") \EV 4
 (string-not-greaterp "Abcde" "abcdE") \EV 5
 (string-lessp "012AAAA789" "01aaab6" :start1 3 :end1 7
                                      :start2 2 :end2 6) \EV 6
 (string-not-equal "AAAA" "aaaA") \EV \term{false}
\endcode

\label Side Effects:\None.

\label Affected By:\None.

\label Exceptional Situations:\None.

\label See Also::

\funref{char=}

\label Notes::

%% 18.2.0 3
\funref{equal} calls \funref{string=} if applied to two \term{strings}.

\endcom

%%% ========== STRINGP
\begincom{stringp}\ftype{Function}

\label Syntax::

\DefunWithValues stringp {object} {generalized-boolean}

\label Arguments and Values::

\param{object}---an \term{object}.

\param{generalized-boolean}---a \term{generalized boolean}.

\label Description::

%% 6.2.2 18
\TypePredicate{object}{string}

\label Examples::

\code
 (stringp "aaaaaa") \EV \term{true}
 (stringp #\\a) \EV \term{false}
\endcode

\label Affected By:\None.

\label Exceptional Situations:\None!

\label See Also::

\funref{typep},
\typeref{string} (\term{type})

\label Notes::

\code
 (stringp \param{object}) \EQ (typep \param{object} 'string)
\endcode

\endcom

%%% ========== MAKE-STRING
\begincom{make-string}\ftype{Function}

\label Syntax::

\DefunWithValues make-string {size {\key} initial-element element-type} {string}

\label Arguments and Values::

\issue{ARGUMENTS-UNDERSPECIFIED:SPECIFY}
\param{size}---a \term{valid array dimension}.
\endissue{ARGUMENTS-UNDERSPECIFIED:SPECIFY}

\issue{ARGUMENTS-UNDERSPECIFIED:SPECIFY}
\param{initial-element}---a \term{character}.
\endissue{ARGUMENTS-UNDERSPECIFIED:SPECIFY}
 \Default{\term{implementation-dependent}}

\issue{CHARACTER-PROPOSAL:2-3-6}
\param{element-type}---a \term{type specifier}.
 \Default{\typeref{character}}
\endissue{CHARACTER-PROPOSAL:2-3-6}

\param{string}---a \term{simple string}.

\label Description::

%% 18.3.0 3
\funref{make-string} returns a \term{simple string} of length \param{size}
whose elements have been initialized to \param{initial-element}.

\issue{CHARACTER-PROPOSAL:2-3-6}
The \param{element-type} names the \term{type} of the \term{elements} of the \term{string};
a \term{string} is constructed of the most \term{specialized}
\term{type} that can accommodate \term{elements} of the given \term{type}.  
\endissue{CHARACTER-PROPOSAL:2-3-6}

\label Examples::

\code
 (make-string 10 :initial-element #\\5) \EV "5555555555"
 (length (make-string 10)) \EV 10
\endcode

\label Affected By::

The \term{implementation}.

\label Exceptional Situations:\None.

\label See Also:\None.

%% Per X3J13. -kmp 05-Oct-93
\label Notes:\None.                                                    

\endcom

%% 18.3.0 4
%% was left out
